\documentclass[]{article}
\usepackage{lmodern}
\usepackage{amssymb,amsmath}
\usepackage{ifxetex,ifluatex}
\usepackage{fixltx2e} % provides \textsubscript
\ifnum 0\ifxetex 1\fi\ifluatex 1\fi=0 % if pdftex
  \usepackage[T1]{fontenc}
  \usepackage[utf8]{inputenc}
\else % if luatex or xelatex
  \ifxetex
    \usepackage{mathspec}
  \else
    \usepackage{fontspec}
  \fi
  \defaultfontfeatures{Ligatures=TeX,Scale=MatchLowercase}
\fi
% use upquote if available, for straight quotes in verbatim environments
\IfFileExists{upquote.sty}{\usepackage{upquote}}{}
% use microtype if available
\IfFileExists{microtype.sty}{%
\usepackage{microtype}
\UseMicrotypeSet[protrusion]{basicmath} % disable protrusion for tt fonts
}{}
\usepackage[margin=1in]{geometry}
\usepackage{hyperref}
\hypersetup{unicode=true,
            pdftitle={Homework 7},
            pdfauthor={Noah Johnson},
            pdfborder={0 0 0},
            breaklinks=true}
\urlstyle{same}  % don't use monospace font for urls
\usepackage{graphicx,grffile}
\makeatletter
\def\maxwidth{\ifdim\Gin@nat@width>\linewidth\linewidth\else\Gin@nat@width\fi}
\def\maxheight{\ifdim\Gin@nat@height>\textheight\textheight\else\Gin@nat@height\fi}
\makeatother
% Scale images if necessary, so that they will not overflow the page
% margins by default, and it is still possible to overwrite the defaults
% using explicit options in \includegraphics[width, height, ...]{}
\setkeys{Gin}{width=\maxwidth,height=\maxheight,keepaspectratio}
\IfFileExists{parskip.sty}{%
\usepackage{parskip}
}{% else
\setlength{\parindent}{0pt}
\setlength{\parskip}{6pt plus 2pt minus 1pt}
}
\setlength{\emergencystretch}{3em}  % prevent overfull lines
\providecommand{\tightlist}{%
  \setlength{\itemsep}{0pt}\setlength{\parskip}{0pt}}
\setcounter{secnumdepth}{0}
% Redefines (sub)paragraphs to behave more like sections
\ifx\paragraph\undefined\else
\let\oldparagraph\paragraph
\renewcommand{\paragraph}[1]{\oldparagraph{#1}\mbox{}}
\fi
\ifx\subparagraph\undefined\else
\let\oldsubparagraph\subparagraph
\renewcommand{\subparagraph}[1]{\oldsubparagraph{#1}\mbox{}}
\fi

%%% Use protect on footnotes to avoid problems with footnotes in titles
\let\rmarkdownfootnote\footnote%
\def\footnote{\protect\rmarkdownfootnote}

%%% Change title format to be more compact
\usepackage{titling}

% Create subtitle command for use in maketitle
\newcommand{\subtitle}[1]{
  \posttitle{
    \begin{center}\large#1\end{center}
    }
}

\setlength{\droptitle}{-2em}
  \title{Homework 7}
  \pretitle{\vspace{\droptitle}\centering\huge}
  \posttitle{\par}
\subtitle{Statistical Inference II}
  \author{Noah Johnson}
  \preauthor{\centering\large\emph}
  \postauthor{\par}
  \predate{\centering\large\emph}
  \postdate{\par}
  \date{April 19, 2018}


\begin{document}
\maketitle

In this homework, the data based on a sample of n = 176 chidren within J
= 10 schools in the American subsample of the PISA (Programme for
International Student Assessment) and is available in PISASchools10.sav.
It has intentionally been provided in an .sav format so you will have to
find the function to read a .sav file.

First, we will look at the relationship between student's home education
resources (HEDRES) on math achievement scores (MATHSCOR) across these 10
schools. Then, we will examine whether the schools' press for academic
excellence (ACADPRES) moderates this relationship. Read each question
carefully, as there are multiple parts to most questions.

\emph{HEDRES} scaled from 1 to 8; lower values indicate poor home
resources for education

\emph{MATHSCOR} average of 50; SD of 9 points

\emph{ACADPRES} scaled from 1 to 8; lower values indicate low press for
academic excellence

\begin{enumerate}
\def\labelenumi{\arabic{enumi}.}
\item
  Use the \emph{sav} data file to run separate multiple regression
  models for each of the 10 schools (X = \emph{hedres}; Y =
  \emph{mathscor}). Review the separate regression results for each of
  the schools. What do you notice about the results (look at the
  correlations between \emph{mathscor} and \emph{hedres}, and the
  regression coefficients for each of the 10 schools)? Is it reasonable
  to assume that the effect of home resources on math achievement is the
  same in all 10 schools? Why or why not? What does your answer imply
  regarding how to include the \emph{hedres} variable within our HLM?
\item
  MODEL 1: Using lme4 fit an unconditional random-effects ANOVA (i.e.,
  \emph{empty model}) with \emph{mathscor} as the outcome. Report and
  interpret all the parameters as well as the ICC.
\item
  MODEL 2: Run a \emph{random coefficients model} with \emph{hedres}
  (\textbf{group-mean centered}) as the predictor of math achievement.
  Report and interpret all the parameters. \textbf{Compared to Model 1},
  how much was the within-schools variability (\(s^2\)) reduced with the
  addition of the group-centered home resources variable?
\item
  Based on your results for step 3, would it make sense to eliminate the
  random effect for the home-resources slope (\(u_1\))? Why or why not?
  Justify your decision.
\item
  MODEL 3: Finally, run a \emph{contextual or conditional model}, and
  add the school academic press (\textbf{centered at the grand-mean}) as
  a level-2 predictor of both the level-1 intercepts and the
  home-resources slopes. Report and interpret all the parameters.
  \textbf{Compared to Model 2}, was the variability in the intercepts
  between schools reduced with the addition of the academic press
  variable? What was the proportion reduction in this variance? What
  about the variability in home-resources slopes? Compared to Model 2,
  what proportion of this variance was accounted for by academic press?
\end{enumerate}


\end{document}
